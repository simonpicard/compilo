\documentclass[a4paper,10pt]{article}
\usepackage[utf8]{inputenc}
\usepackage[T1]{fontenc}
\usepackage[francais]{babel}
\usepackage[a4paper]{geometry}
\geometry{hscale=0.75,vscale=0.60,centering}

%opening
\title{INFO-F-403 : Language theory and compiling \\ Rapport projet partie 1}
\author{Simon Picard \\ Arnaud Rosette}

\begin{document}

\maketitle


\section{Expressions régulières}

Afin de décrire les mots (tokens) acceptés par le langage Iulius, nous utilisons les expressions régulières étendues.

Voici le tableau qui reprend l'ensemble des tokens du langage Iulius. Chaque ligne de celui-ci est composée du nom du token, le type dans l'enum LexicalUnit qui lui est associé et de l'expression régulière qui correspond à ce token.

\begin{tabular}{|l|c|r|}
  \hline
  Nom & Type & Expression régulière \\
  \hline
  Comment & / & \begin{minipage}{2in} \begin{verbatim} #.*(\\r)?\\n \end{verbatim} \end{minipage} \\
  Circumflex ( $\widehat{}$ ) & POWER & \begin{minipage}{2in} \begin{verbatim} \^ \end{verbatim} \end{minipage} \\
  Percentage (\%) & REMAINDER & \begin{minipage}{2in} \begin{verbatim} % \end{verbatim} \end{minipage} \\
  Tilde (\textasciitilde) & BITWISE\_NOT & \begin{minipage}{2in} \begin{verbatim} ~ \end{verbatim} \end{minipage} \\
  Pipe (|) & BITWISE\_OR & \begin{minipage}{2in} \begin{verbatim} \| \end{verbatim} \end{minipage} \\
  Dollar (\$) & BITWISE\_XOR & \begin{minipage}{2in} \begin{verbatim} \$ \end{verbatim} \end{minipage} \\
  Dubble greater (>>) & ARITHMETIC\_SHIFT\_RIGHT & \begin{minipage}{2in} \begin{verbatim} >> \end{verbatim} \end{minipage} \\
  Dubble lower (<<) & ARITHMETIC\_SHIFT\_LEFT & \begin{minipage}{2in} \begin{verbatim} << \end{verbatim} \end{minipage} \\
  Dubble equal (==) & EQUALITY & \begin{minipage}{2in} \begin{verbatim} == \end{verbatim} \end{minipage} \\
  Exclamation equal (!=) & INEQUALITY & \begin{minipage}{2in} \begin{verbatim} != \end{verbatim} \end{minipage} \\
  Function & FUNCTION & \begin{minipage}{2in} \begin{verbatim} function \end{verbatim} \end{minipage} \\
  Return & RETURN & \begin{minipage}{2in} \begin{verbatim} return \end{verbatim} \end{minipage} \\
  Arrow right (->) & MAP\_TO & \begin{minipage}{2in} \begin{verbatim} -> \end{verbatim} \end{minipage} \\
  Question mark (?) & TERNARY\_IF & \begin{minipage}{2in} \begin{verbatim} \? \end{verbatim} \end{minipage} \\
  Exclamation mark (!) & NEGATION & \begin{minipage}{2in} \begin{verbatim} ! \end{verbatim} \end{minipage} \\
  Colon (:) & TERNARY\_ELSE & \begin{minipage}{2in} \begin{verbatim} : \end{verbatim} \end{minipage} \\
  Dubble ampersand (\&\&) & LAZY\_AND & \begin{minipage}{2in} \begin{verbatim} && \end{verbatim} \end{minipage} \\
  Dubble pipe (||) & LAZY\_OR & \begin{minipage}{2in} \begin{verbatim} \|\| \end{verbatim} \end{minipage} \\
  While & WHILE & \begin{minipage}{2in} \begin{verbatim} while \end{verbatim} \end{minipage} \\
  For & FOR & \begin{minipage}{2in} \begin{verbatim} for \end{verbatim} \end{minipage} \\
  Semicolon (;) & END\_OF\_INSTRUCTION & \begin{minipage}{2in} \begin{verbatim} ; \end{verbatim} \end{minipage} \\
  Println & PRINTLN & \begin{minipage}{2in} \begin{verbatim} println \end{verbatim} \end{minipage} \\
  Const & CONST & \begin{minipage}{2in} \begin{verbatim} const \end{verbatim} \end{minipage} \\
  Let & LET & \begin{minipage}{2in} \begin{verbatim} let \end{verbatim} \end{minipage} \\
  Dubble colon (::) & TYPE\_DEFINITION & \begin{minipage}{2in} \begin{verbatim} :: \end{verbatim} \end{minipage} \\
  Boolean (type)  & BOOLEAN\_TYPE & \begin{minipage}{2in} \begin{verbatim} Bool \end{verbatim} \end{minipage} \\
  Real (type) & REAL\_TYPE & \begin{minipage}{2in} \begin{verbatim} FloatingPoint \end{verbatim} \end{minipage} \\
  Integer (type) & INTEGER\_TYPE & \begin{minipage}{2in} \begin{verbatim} Integer \end{verbatim} \end{minipage} \\
  Integer (cast) & INTEGER\_CAST & \begin{minipage}{2in} \begin{verbatim} int \end{verbatim} \end{minipage} \\
  Real (cast) & REAL\_CAST & \begin{minipage}{2in} \begin{verbatim} float \end{verbatim} \end{minipage} \\
  Read integer & READ\_INTEGER & \begin{minipage}{2in} \begin{verbatim} readint \end{verbatim} \end{minipage} \\
  Read real & READ\_REAL & \begin{minipage}{2in} \begin{verbatim} readfloat \end{verbatim} \end{minipage} \\
  Boolean (cast) & BOOLEAN\_CAST & \begin{minipage}{2in} \begin{verbatim} bool \end{verbatim} \end{minipage} \\
  \hline
\end{tabular}



\end{document}
